% Options for packages loaded elsewhere
\PassOptionsToPackage{unicode}{hyperref}
\PassOptionsToPackage{hyphens}{url}
%
\documentclass[
]{article}
\usepackage{amsmath,amssymb}
\usepackage{lmodern}
\usepackage{iftex}
\ifPDFTeX
  \usepackage[T1]{fontenc}
  \usepackage[utf8]{inputenc}
  \usepackage{textcomp} % provide euro and other symbols
\else % if luatex or xetex
  \usepackage{unicode-math}
  \defaultfontfeatures{Scale=MatchLowercase}
  \defaultfontfeatures[\rmfamily]{Ligatures=TeX,Scale=1}
\fi
% Use upquote if available, for straight quotes in verbatim environments
\IfFileExists{upquote.sty}{\usepackage{upquote}}{}
\IfFileExists{microtype.sty}{% use microtype if available
  \usepackage[]{microtype}
  \UseMicrotypeSet[protrusion]{basicmath} % disable protrusion for tt fonts
}{}
\makeatletter
\@ifundefined{KOMAClassName}{% if non-KOMA class
  \IfFileExists{parskip.sty}{%
    \usepackage{parskip}
  }{% else
    \setlength{\parindent}{0pt}
    \setlength{\parskip}{6pt plus 2pt minus 1pt}}
}{% if KOMA class
  \KOMAoptions{parskip=half}}
\makeatother
\usepackage{xcolor}
\IfFileExists{xurl.sty}{\usepackage{xurl}}{} % add URL line breaks if available
\IfFileExists{bookmark.sty}{\usepackage{bookmark}}{\usepackage{hyperref}}
\hypersetup{
  pdftitle={Modelización de serie de tiempo de precios mayoristas de manzana de Uruguay.},
  pdfauthor={Emanuelle Marsella, Maximiliano Saldaña},
  hidelinks,
  pdfcreator={LaTeX via pandoc}}
\urlstyle{same} % disable monospaced font for URLs
\usepackage[margin=1in]{geometry}
\usepackage{graphicx}
\makeatletter
\def\maxwidth{\ifdim\Gin@nat@width>\linewidth\linewidth\else\Gin@nat@width\fi}
\def\maxheight{\ifdim\Gin@nat@height>\textheight\textheight\else\Gin@nat@height\fi}
\makeatother
% Scale images if necessary, so that they will not overflow the page
% margins by default, and it is still possible to overwrite the defaults
% using explicit options in \includegraphics[width, height, ...]{}
\setkeys{Gin}{width=\maxwidth,height=\maxheight,keepaspectratio}
% Set default figure placement to htbp
\makeatletter
\def\fps@figure{htbp}
\makeatother
\setlength{\emergencystretch}{3em} % prevent overfull lines
\providecommand{\tightlist}{%
  \setlength{\itemsep}{0pt}\setlength{\parskip}{0pt}}
\setcounter{secnumdepth}{5}
\newlength{\cslhangindent}
\setlength{\cslhangindent}{1.5em}
\newlength{\csllabelwidth}
\setlength{\csllabelwidth}{3em}
\newlength{\cslentryspacingunit} % times entry-spacing
\setlength{\cslentryspacingunit}{\parskip}
\newenvironment{CSLReferences}[2] % #1 hanging-ident, #2 entry spacing
 {% don't indent paragraphs
  \setlength{\parindent}{0pt}
  % turn on hanging indent if param 1 is 1
  \ifodd #1
  \let\oldpar\par
  \def\par{\hangindent=\cslhangindent\oldpar}
  \fi
  % set entry spacing
  \setlength{\parskip}{#2\cslentryspacingunit}
 }%
 {}
\usepackage{calc}
\newcommand{\CSLBlock}[1]{#1\hfill\break}
\newcommand{\CSLLeftMargin}[1]{\parbox[t]{\csllabelwidth}{#1}}
\newcommand{\CSLRightInline}[1]{\parbox[t]{\linewidth - \csllabelwidth}{#1}\break}
\newcommand{\CSLIndent}[1]{\hspace{\cslhangindent}#1}
\usepackage{float}
\usepackage[spanish]{babel}
\usepackage{amsmath}
\ifLuaTeX
  \usepackage{selnolig}  % disable illegal ligatures
\fi

\title{Modelización de serie de tiempo de precios mayoristas de manzana
de Uruguay.}
\author{Emanuelle Marsella, Maximiliano Saldaña}
\date{Junio 2022}

\begin{document}
\maketitle

\hypertarget{resumen-ejecutivo}{%
\section{Resumen ejecutivo}\label{resumen-ejecutivo}}

\hypertarget{anuxe1lisis-descriptivo}{%
\section{Análisis descriptivo}\label{anuxe1lisis-descriptivo}}

La serie a ser estudiada es la de precios promedio mensuales del kilo de
manzana en la Unidad Agroalimentaria Metropolitana (ex Mercado Modelo).
Los precios de los distintos rubros transados en este mercado mayorista
de frutas y hortalizas son relevados por el Observatorio Granjero dos
veces a la semana, los lunes y los jueves, mediante encuestas a los
distintos vendedores informantes. Se relevan precios por distintas
variedades, calidades y calibres. Empleando los distintos precios
obtenidos los técnicos del Observatorio llegan a un precio de referencia
por consenso.

Se cuentan con los datos desde enero de 2013 a mayo de 2022 y se
considerará el promedio mensual de los precios, por lo que se cuentan
con 113 observaciones. En lugar de emplear los datos bisemanales o el
promedio semanal se opta por la frecuencia mensual debido a la
dificultad de emplear el herramental de los modelos SARIMA para tales
tipos de series, en particular para el tratamiento de la estacionalidad.

\begin{figure}

{\centering \includegraphics[width=0.8\linewidth]{informe_files/figure-latex/plot_precios-1} 

}

\caption{Serie de precios mensuales del Kg de manzana en pesos Uruguayos.}\label{fig:plot_precios}
\end{figure}

En la figura @ref(fig:plot\_precios) se presenta el gráfico de la serie
a ser trabajada. La impresión inicial que da es que la serie presenta
cierto patrón estacional anual, donde los precios comienzan altos para
luego descender hasta el segundo trimeste de los años y luego tienden a
elevarse hasta el final de año. Esto se puede observar mejor en el
gráfico de los precios coloreados por año y el gráfico de la evolución
de los precios año a año por mes de la figura
@ref(fig:plot\_precios\_seas). El año 2020 presenta precios atípicamente
altos y un comportamiento marcadamente distinto al de los otros años, no
se observa la caída inicial de precios sino un aumento sostenido. Esto
se puede deber al impacto económico que causó la pandemia de
Coronavirus, que llegó a nuestro país en dicho año. Ya para 2021 y lo
que va de 2022 parece haber una vuelta a patrones previos. Todo esto
deberá ser tenido en cuenta a la hora de la especificación de un modelo
del tipo ARIMA/SARIMA.

\begin{figure}

{\centering \includegraphics[width=0.8\linewidth]{informe_files/figure-latex/plot_precios_seas-1} 

}

\caption{Serie de precios mensuales del Kg de manzana en pesos Uruguayos.}\label{fig:plot_precios_seas}
\end{figure}

\begin{center}\includegraphics[width=0.8\linewidth]{informe_files/figure-latex/descomp-1} \end{center}

Para ahondar en el análisis descriptivo se realiza la descomposición de
la serie en tendencia/ciclo, estacionalidad y componente irregular. En
la figura @ref(fig:descomp) se presentan las series de los componentes
resultado de una descomposición mediante \emph{STL} (Seasonal Trend
Descomposition using LOESS) por separado.

Se puede apreciar una marcada estacionalidad anual y en los últimos 6
años un ciclo corto que se repite cada dos años. La fuerza de la
estacionalidad, definida como (Hyndman and Athanasopoulos 2018):

\[F_s = max\left(0, 1-\frac{Var(R_t)}{Var(R_t+ S_t)}\right)\] toma el
valor 0.51 (entre más cercano a 1 más fuerte el componente). Esto
refuerza la necesidad de considerar la estacionalidad a la hora de
especificar un modelo.

\hypertarget{metodologuxeda-y-resultados}{%
\section{Metodología y resultados}\label{metodologuxeda-y-resultados}}

\hypertarget{muestra-de-entrenamiento-y-de-prueba}{%
\subsection{Muestra de entrenamiento y de
prueba}\label{muestra-de-entrenamiento-y-de-prueba}}

Resulta de interés que el modelo ajustado a la serie sea de utilidad
para la predicción. Para poder evaluar la calidad de las predicciones,
una manera que busca replicar el proceso de obtención de nuevos datos es
dividir la serie en una muestra de entrenamiento y una de prueba. La
primera se emplea para ajustar el modelo, a partir del cual se
realizarán las predicciones. Se dejan las últimas 12 observaciones para
la muestra de prueba, que son los precios que van desde junio de 2021 a
mayo de 2022. Debe tenerse en cuenta que el periodo del final de la
muestra de entrenamiento y también la muestra de prueba están enmarcados
en el contexto de gran incertidumbre que presenta la pandemia, por lo
que deberá tenerse especial cuidado con el tratamiento de atípicos y las
conclusiones que se tomen sobre las predicciones.

\hypertarget{identificaciuxf3n}{%
\subsection{Identificación}\label{identificaciuxf3n}}

La primera fase para el modelado ARIMA de una serie de tiempo en el
marco de la metodología de Box-Cox es la identificación del modelo, que
consiste en determinar en un principio en detectar la estructura de
autocorrelación, la cantidad de parámetros con la que contará la
especificación, si la serie necesita diferenciación y si resultará
necesaria alguna otra transformación.

\hypertarget{transformaciuxf3n-logaruxedtimica}{%
\subsubsection{Transformación
logarítimica}\label{transformaciuxf3n-logaruxedtimica}}

La transformación logarítmica de una serie de tiempo puede tener como
resultado una reducción del error de predicción en el caso de que
estabilice la varianza (Lütkepohl and Xu 2009). Esto se cumple en
particular cuando la varianza aumenta con la media de la serie, lo cual
no es el caso de los precios de manzana, que si bien presentan una
varianza que aumenta en el tiempo no parece haber una tendencia
creciente clara. Por lo tanto, esta transformación no resultaría
aconsejable de aplicar.

Para confirmar esto, se considera la transformación de Box-Cox, donde
siendo \(y\) la variable transformada y \(x\) la variable a transformar:

\[y_t = \begin{cases} \frac{x_t - 1}{\lambda} \,\,\, si \,\,\, \lambda \ne 0 \\ \ln x_t \,\,\, si \,\,\, \lambda = 0 \end{cases}\]

Donde el parámetro \(\lambda\) se estima por máxima verosimilitud. En el
caso de la serie planteada, dicho parámetro toma el valor -0.59, por lo
que la transformación logarítmica no resulta adecuada.

\hypertarget{autocorrelaciuxf3n-y-autocorrelaciuxf3n-parcial}{%
\subsubsection{Autocorrelación y autocorrelación
parcial}\label{autocorrelaciuxf3n-y-autocorrelaciuxf3n-parcial}}

Como primer elemento a considerar en la identificación de un modelo
ARIMA resultan útiles las funciones de autocorrelación y autocorrelación
parcial. Al graficarlas en ellas se puede visualizar la estructura de
dependencia temporal de la serie trabajada e indicios de la
estacionariedad o ausencia de ella. La función de autocorrelación (ACF)
se define como:

\[\rho(t, t+j) = \frac{Cov(Y_t, Y_{t+j})}{\sigma_t\sigma_{t+j}}  \]

Mientras que la función de autocorrelación parcial (PACF) es la
autocorrelación entre \(Y_t\) y \(Y_{t+j}\) una vez se quita el efecto
de las correlaciones intermedias que hay entre ambas variables (Collazo
2022).

Estas funciones se deben estimar a partir de los datos, obteniéndose la
autocorrelación muestral. En la figura @ref(fig:acf1) del Anexo se
presentan los primeros 72 valores de ambas funciones para la serie
original de precios de manzana, junto con el intervalo de confianza de
la prueba de hipótesis \(H_0) \rho_j = 0\) vs \(H_0) \rho_j \ne 0\). En
la ACF se puede distinguir que las primeras 5 autocorrelaciones resultan
significativas y presentan un decaimiento exponencial y que también son
significativas las que están en torno al rezago 24 y entre los rezagos
36 y 48. Lo primero puede ser indicio de una estructura autorregresiva
subyacente y lo segundo un indicio de estacionalidad anual o bianual.

Por otro lado, en la PACF se aprecia que los valores para los primeros
dos lags son significativamente distintos a 0. Esto puede ser un indicio
de que hay una estructura autorregresiva, posiblemente de segundo orden,
en los datos.

\hypertarget{dominio-de-las-frecuencias}{%
\subsubsection{Dominio de las
frecuencias}\label{dominio-de-las-frecuencias}}

Desde la perspectiva del dominio de las frecuencias de la serie se
considera esta última en su expresión trigonométrica, mediante una suma
ponderada de funciones periódicas coseno y seno. El espectro poblacional
puede resultar de utilizad para observar la estructura de variabilidad
de la serie, dado que el área por debajo del mismo es la variabilidad
asociada a las frecuencias consideradas.

En la figura @ref(fig:espectro) se presenta la estimación no paramétrica
del espectro poblacional de la serie de precios de manzana trabajada. En
esta estimación se hace uso del periodograma muestral, que es la
estimación del espectro poblacional a partir de las autocovarianzas
muestrales y luego se realiza un promedio ponderado de sus valores
mediante un \emph{kernel} a fines de suavizar el resultado, que en
general resulta difícil de interpretar inicialmente. En este caso se
pondera con el \emph{kernel} de Daniell modificado ponderando de a 3
valores del periodograma muestral 2 veces sucesivas. Se puede apreciar
como las frecuencias menores, aquellas asociadas a periodos más largos
(teniendo en cuenta que \(p =2\pi/w\), siendo \(p\) el periodo y \(w\)
la frecuencia) son aquellas que acumulan mayor variabilidad. Esto puede
considerarse como otro indicio de una dependencia temporal estacional
entre las observaciones de la muestra.

\hypertarget{tests-de-rauxedces-unitarias}{%
\subsubsection{Tests de raíces
unitarias}\label{tests-de-rauxedces-unitarias}}

Dado que las funciones de autocorrelación dieron indicios de que el
proceso no es estacionario, resulta de interés poner a prueba si el
proceso es \(I(1)\), es decir, cuenta con una raíz unitaria. En dicho
caso el proceso es no estacionario la cual implica que no puede ser
modelado en el marco de los ARMA.

Hay múltiples pruebas de hipótesis que han sido desarrolladas con el
propósito de identificar raíces unitarias. Dos de ellos son el de
Dickey-Fuller aumentado y el de Phillips-Perron. En el caso del primero
se especifica el proceso proceso estocástico subyacente como:

\[Y_t = \rho Y_{t-1} + \varepsilon_t, \,\,\,\, \varepsilon_t \stackrel{\text{iid}}{\sim} N(0, \sigma^2)\]
Y se contrasta:

\[
H_0) \rho = 1 \\
H_1) \rho < 1 \\
\]

Empleando el estadístico \$ (\hat{\rho} - 1) / \sigma\_\{\hat{\rho}\}
\$, que si \$ \rho \$ es menor a 1 en valor absoluto se distribuye
normal asintóticamente y si es igual a 1 debe usarse una distribución
empírica tabulada por Fuller. El test además toma en cuenta la posible
autocorrelación de los errores incluyendo rezagos de la variable en la
regresión auxiliar para mayor robustez.

Por otro lado la propuesta de prueba de raíz unitaria de Phillip-Perron
se basa en la de Dickey y Fuller, pero además los compatibilizan con la
presencia de heteroscedasticidad y/o autocorrelación de los errores.

Para la serie planteada, al observar los p-valores, ambos tests llevan a
no rechazar la hipótesis de raíz unitaria al 95\% de confianza. Esto
lleva a concluir que una primera diferencia resulta necesaria para
llevar la serie a la estacionariedad.

\begin{figure}

{\centering \includegraphics[width=0.8\linewidth]{informe_files/figure-latex/acf2-1} 

}

\caption{Autocorrelograma y autocorrelograma parcial de la serie de primeras difernecias de precios de manzana.}\label{fig:acf2}
\end{figure}

Observando los nuevos autocorrelogramas (@ref(fig:acf2) se puede notar
como de entre los primeros valores de la ACF solo el primero resulta
significativamente distinto a 0 (lo que puede indicar un componente de
medias móviles de primer orden) y lo mismo se da para los valores en
torno al lag 24 y 48. Esto último vuelve a indicar la presencia de una
estructura de dependencia estacional (indicio de un componente
estacional autorregresivo de orden 2). Por otro lado, en la PACF, se
aprecia que solo los dos primeros valores son significativos, lo cual
puede indicar cierta estructura autorregresiva, posiblemente de orden 2.
A pesar de esto, hay que tener en cuenta que en las aplicaciones
prácticas al tratar de identificar el orden de un proceso ARIMA mediante
la FAC y PACF la distinción entre especificaciones puede volverse difusa
y no resulta tan claro la cantidad de parámetros a elegir.

Teniendo en cuenta este acercamiento metodológico, inicialmente se
plantea un modelo \(ARIMA(2,1,1)(2,0,0)\) (en adelante ``Modelo
manual'').

\hypertarget{selecciuxf3n-mediante-criterios-de-informaciuxf3n}{%
\subsubsection{Selección mediante criterios de
información}\label{selecciuxf3n-mediante-criterios-de-informaciuxf3n}}

Una forma alternativa de elegir la especificación del modelo, esto es,
la cantidad de parámetros y de diferencias, es mediante los criterios de
información. Con estos criterios se busca elegir entre una selección de
modelos aquel con una mayor verosimilitud, penalizando por la cantidad
de parámetros que tenga.

Uno de los más empleados y en particular el que se considera en la
función \emph{auto.arima()} empleada para la estimación es el criterio
de información de Akaike corregido (AICc). Su fórmula para un modelo es:

\[ AICc = T \log{\hat{\sigma}^2_{MV}} + T \frac{1+k/T}{1-(k+2)/T} \]
Siendo \(T\) el número de observaciones, \(k\) el de parámetros y \$
\hat{\sigma}\^{}2\_\{MV\} \$ el estimador máximo verosímil de la
varianza de los errores. El modelo seleccionado es aquel con el menor
valor del AICc, lo cuál a nivel algorítmico se hace empleando la
selección \emph{stepwise} a partir de un conjunto inicial de modelos
(Hyndman and Athanasopoulos 2018).

El modelo seleccionado mediante esta metodología (de ahora en adelante
``Modelo AICc'') resulta en un \(ARIMA(0,1,2)(0,0,2)\). Se puede
observar que no se considera un componente autorregresivo, pero si un
componente estacional de medias móviles y al igual que el anterior
modelo, una diferencia (el algoritmo de \emph{auto.arima()} realiza
múltiples tests de raíz unitaria). Además, se considera un componente
estacional de medias móviles de orden 2, en lugar de autorregresivo como
era en el caso del modelo anterior.

\hypertarget{estimaciuxf3n}{%
\subsection{Estimación}\label{estimaciuxf3n}}

Luego de haber identificado la especificación del modelo, el paso que
sigue es estimar sus parámetros, dado que no es posible conocer sus
verdaderos valores al ser una construcción teórica. Para esto se recurre
a la estimación por máxima verosimilitud, donde las estimaciones
obtenidas son las que maximizan la probabilidad de que se haya observado
la muestra con la que se cuenta. Es necesario asumir que los errores son
gaussianos, un supuesto fuerte pero las estimaciones resultantes que
emanen a partir de hacerlo serán razonables aunque no se cumpla
(Hamilton 1994).

En particular, el método empleado es la estimación máximo verosímil
condicional, donde se supone que la primera observación de la serie es
determinística y se maximiza la verosimilitud condicionada a dicha
observación. Esto simplifica las expresiones de las funciones, y si el
tamaño de muestra es razonablemente grande, la primera observación no
tendrá gran efecto sobre la verosimilitud estimada.

\hypertarget{pruebas-de-significaciuxf3n-de-los-paruxe1metros}{%
\subsubsection{Pruebas de significación de los
parámetros}\label{pruebas-de-significaciuxf3n-de-los-paruxe1metros}}

Para completar la especificación se realizan pruebas de hipótesis sobre
la significación de los parámetros estimados. La hipótesis para cada uno
de ellos, siendo \(\lambda\) un parámetro cualquiera del modelo:

\[H_0) \lambda = 0\] \[H_1)\lambda \ne 0\] Donde el estadístico empleado
es:

\[z = \hat{\lambda} / T\hat{\sigma}_{\lambda}\] Para el que se cumplirá
la Normalidad asintótica debido a que la estimación se realizó por
máxima verosimilitud (Hamilton 1994, 143).

Realizando esta prueba para los coeficientes del modelo manual todos
resultan significativos al 5\% de confianza con la excepción del
parámetro de medias móviles (MA) y el primer parámetro autorregresivo
estacional (sAR). Se opta entonces por eliminar el componente MA de la
especificación y restringir el primer parámetro sAR a 0.

\hypertarget{la-especificaciuxf3n-del-modelo-manual-final}{%
\subsubsection{La especificación del modelo manual
final}\label{la-especificaciuxf3n-del-modelo-manual-final}}

La forma del modelo identificado de manera manual es:

\[\Phi_2(L^{12})\phi_2(L)\Delta^1 Y_t =  \varepsilon_t\] Donde:

\[\Phi_2(L^{12}) = 1-\Phi_2 L^{24}\]

\[\phi_2(L) = 1-\phi_1 L-\phi_2 L^{2}\]

\[\Delta^1 = 1-L\] Y se supone en un principio que los residuos son de
media 0, homoscedásticos, incorrelacionados y:

\[\epsilon_t \stackrel{\text{iid}}{\sim} N(0, \sigma^2) \]

Los parámetros resultantes de la estimación para el modelo manual, son:

\begin{itemize}
\item
  Parte autorregresiva: \$ \hat{\phi}\_1 = 0.557 \$ y \$ \hat{\phi}\_1 =
  -0.277 \$
\item
  Parte medias móviles: \$ \hat{\theta}\_1 = -0.4157 \$
\item
  Parte autorregresiva estacional: \$ \hat{\Phi}\_2 = 0.415 \$
\item
  Varianza de las estimaciones: \$ \hat{\sigma}\^{}2 = 15.82 \$
\end{itemize}

\hypertarget{diagnuxf3stico}{%
\subsection{Diagnóstico}\label{diagnuxf3stico}}

Como se mencionó anteriormente, a simple vista es posible identificar
que la pandemia de Coronavirus tuvo un impacto sobre los precios, así
que esto deberá ser incluido en la modelización cuanto antes. Una manera
de tener en cuenta este efecto atípico es considerar que hay un cambio
transitorio (TC por sus siglas en inglés), un suceso que tiene un efecto
que perdura en la serie pero no es permanente.

Posteriormente se deberá poner a prueba la presencia de otros atípicos.

\hypertarget{contraste-media-0}{%
\subsection{Contraste media 0}\label{contraste-media-0}}

\hypertarget{contraste-homoscedasticidad}{%
\subsection{Contraste
homoscedasticidad}\label{contraste-homoscedasticidad}}

\hypertarget{predicciuxf3n}{%
\section{Predicción}\label{predicciuxf3n}}

Realizaremos predicción para los siguientes modelos, que son los que
mejor cumplen los supuestos:

\begin{itemize}
\item
  \textbf{modelo\_tso:} Modelo intervenido con el procedimiento
  automático de detección de outliers del paquete \emph{tso}. Cumple
  tanto el supuesto de no autocorrelación de los residuos como el
  supuesto de normalidad.
\item
  \textbf{modelo\_clean:} Modelo intervenido con por métodos del paquete
  \emph{tsoutliers}, que reemplaza los mismos a través de interpolación
  lineal. Cumple el supuesto de no autocorrelación de los residuos, pero
  no el supuesto de normalidad.
\item
  \textbf{modelo\_tbats:} Modelo TBATS (explicar más). Cumple el
  supuesto de no autocorrelación de los residuos, pero no el supuesto de
  normalidad (la única hipótesis nula que no se rechaza en este sentido
  es la de asimetría).
\end{itemize}

\hypertarget{conclusiones}{%
\section{Conclusiones}\label{conclusiones}}

\hypertarget{anexo}{%
\section{Anexo}\label{anexo}}

\begin{figure}

{\centering \includegraphics[width=0.8\linewidth]{informe_files/figure-latex/acf1-1} 

}

\caption{Funciones de autocorrelación y autocorrelación parcial muestrales de la serie de precios de manzana.}\label{fig:acf1}
\end{figure}

\hypertarget{refs}{}
\begin{CSLReferences}{1}{0}
\leavevmode\vadjust pre{\hypertarget{ref-notas_series}{}}%
Collazo, Silvia Rodríguez. 2022. \emph{Series Cronológicas: Notas de
Curso}.

\leavevmode\vadjust pre{\hypertarget{ref-hamilton_1994}{}}%
Hamilton, James Douglas. 1994. \emph{Time Series Analysis}. Princeton
University Press.

\leavevmode\vadjust pre{\hypertarget{ref-hyndman2018}{}}%
Hyndman, Robin John, and George Athanasopoulos. 2018. \emph{Forecasting:
Principles and Practice}. 2nd ed. Australia: OTexts.

\leavevmode\vadjust pre{\hypertarget{ref-lutkepohl2009}{}}%
Lütkepohl, Helmut, and Fang Xu. 2009. {``The Role of the Log
Transformation in Forecasting Economic Variables.''} Working paper No.
2591. Munich, Alemania: CESifo.

\end{CSLReferences}

\end{document}
